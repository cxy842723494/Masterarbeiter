\chapter{Zweite Method} \label{cha:ZweiteMethod}

In diesem Kapitel werden die erste Erfahrung beschrieben. Zuerst läuft die Vorstellung des DaVid-Systems. Die Systemmodell und Arbeitsprinzip des Systems werden in anschließenden Abschnitt erläutert. Schließlich folgt die mögliche Anwendungsgebiete des Systems.

Nicht vergessen, dass Überschriften nicht aufeinander folgen dürfen\ldots

\section{Allgemeine Struktur}


\section{Binariesierung}

\textbf{Grundlegende adaptive Schwellenwert Binarisierungsmethode}

Ein Bildgebungsfaktor ungleichmäßiger Helligkeit bewirkt, dass ein Histogramm, das ansonsten für eine effiziente Segmentierung geeignet wäre, ein Histogramm wird, das nicht effektiv mit einem einzigen globalen Schwellenwert segmentiert werden kann.


\textbf{OTSU adaptive Schwelle Methode}

OTSU, auch bekannt als die maximale Interklassenvarianz, wurde 1979 vom japanischen Gelehrten Otsu vorgeschlagen und ist eine adaptive Schwellenwertbestimmungsmethode, die auch als Otsu bekannt ist.

Die Grundidee des Otsu-Algorithmus ist: Für das Bild stellt das T die Segmentierungsschwellenwert zwischen Vordergrund und des Hintergrund dar. Die Anzahl der Vordergrundpunkte ist W0, der durchschnittliche Grauwert ist U0, dagegen die Anzahl der Hintergrundpunkte ist W1 und der durchschnittliche Grauwert ist U1. Dann ist das gesamte durchschnittliche Grauwert des Bildes:




















\section{Morphologie}



\section{Canny detection}



\section{Hough detection}


