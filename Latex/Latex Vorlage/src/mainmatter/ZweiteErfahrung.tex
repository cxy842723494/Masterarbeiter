\chapter{ZweiteErfahrung} \label{cha:ZweiteErfahrung}

In diesem Kapitel wird der durch die Projektgruppe implementierte VXI-11-Server vorgestellt und in den einzelnen Abschnitten auf die Teilbereiche Hardware, Software und auf Kommunikationsschnittstellen mit Test- und Messgeräten eingegangen. Die grundlegende Idee des VXI-11-Servers ist die Kommunikation mit Geräten über Ethernet, welche unterschiedliche Kommunikationsstandards nutzen. Hierzu ist eine entsprechende Hardware notwendig, um verschiedenste Kommunikationsstandards anschließen zu können und eine entsprechende softwaretechnische Implementierung. Auf die Realisierung dieser Teilbereiche wird im folgenden eingegangen.

\section{Grundlegende Einführung} 
\cite{lin1973}.

\section{Bildregistrierung} 

\begin{figure}[htb]
 \centering 
 \includegraphics[keepaspectratio,width=0.8\textwidth]{images/0_Image_Registration_Flussdiagramm.pdf}
 \caption{Flussdiagramm der Bildregistrierung}
 \label{fig:Bildregistrierung}
\end{figure}

\subsection{SURF}
Hier wird zuerste die SURF\cite{Surf} Feature Detektion eingegangen. Der SURF-Algorithmus arbeitet mit integrierten Bildern. Die Faltung bezieht sich nur auf das vorherige Bild, und mit Erhöhung der Größe des Bildkerns können das Heruntertaktung-Verfahren realisiert werden. 

\textbf{Algorithmus:}\\
\\
$\bullet$ \textbf{Aufbau einer hessischen Matrix.}\\
Die Hesse-Matrix stellt den Kern des SURF-Algorithmus dar. Zur Vereinfachung der Operation wird die Funktion f (z, y) angenommen, dass die Hesse-Matrix H setzt sich aus Funktionen und partiellen Ableitungen zusammen:

\begin{equation}
   \mathcal{H}(f(x,y)) = \begin{bmatrix}
   \frac{\partial^{2}f}{\partial x^{2}} & \frac{\partial^{2}f}{\partial x \cdot \partial y} \\
   \frac{\partial^{2}f}{\partial x \cdot \partial y} & \frac{\partial^{2}f}{\partial y^{2}} \\   
   \end{bmatrix}
\end{equation}

 Diskriminante der H-Matrix läuft:
 
\begin{equation}
   \det(\mathcal{H}) = \frac{\partial^{2}f}{\partial x^{2}} \cdot \frac{\partial^{2}f}{\partial y^{2}} - (\frac{\partial^{2}f}{\partial x \cdot \partial y})^2  
\end{equation}

Der Wert der Diskriminante ist der Eigenwert der H-Matrix. Durch dessen positiven und negativen wird bestimmt, ob der Punkt ein Extrempunkt ist oder nicht. Im SURF Algorithmus wird das Bildpixel $l(x,y)$ anstelle des Funktionswertes $f(x,y)$ verwendet. Nutzen eine Zweite-Order Gaussian function als Filter. Die zweiten Partielle Ableitungen können durch Faltung zwischen bestimmten Kernen berechnet werden. Dadurch können die Werte der drei Matrixelemente der H-Matrix berechnet werden, d.h. die H-Matrix berechnet:

\begin{equation}
\begin{split}
   &\mathcal{H}(\textbf{x},\sigma) = \begin{bmatrix}
   L_{xx}(\textbf{x},\sigma)\ L_{xy}(\textbf{x},\sigma) \\
   L_{xy}(\textbf{x},\sigma)\ L_{yy}(\textbf{x},\sigma)
   \end{bmatrix} \\   
   &L(\textbf{x},\sigma) = G(\sigma)*I(\textbf{x}) \\  
   &G(\sigma) = \frac{\partial^{2}g(\sigma)}{\partial x^{2}}      
\end{split}
\end{equation}


Hier $L_{xx}(\textbf{x},\sigma)$ bedeutet die Faltung der zweiter Gaussian Ableitung $G(\sigma)$ mit dem Bild I in Punkt $\textbf{x}$(x,y), ähnlich für $L_{xy}(\textbf{x},\sigma)$ und $L_{yy}(\textbf{x},\sigma)$. Auf diese Weise kann der Wert der Determinante für jedes Pixel in dem Bild berechnet werden, und dieser Wert kann verwendet werden, um den Merkmalspunkt zu feststellen.
Zur einfacheren Anwendung schlägt Herbert Bay\cite{Surf} vor, L mit einer Approximation ersetzen. Um den Fehler zwischen dem genauen Wert und der Approximation auszugleichen, kann die H-Matrix-Diskriminante wie folgt ausgedrückt werden:

\begin{equation}
   \det(\mathcal{H}_{Approx}) = D_{xx}D_{yy} - (0.9D_{xy})^2  
\end{equation}
\\
$\bullet$ \textbf{Erstellen Maßstab Raum}\\
Der Maßstabsraum $L(\textbf{x},\sigma)$ des Bildes ist die Darstellung dieses Bildes bei unterschiedlichen Auflösungen(Skalierung). Im Bereich der Computer Vision wird der Maßstabsraum symbolisch als Bildpyramide ausgedrückt. Wobei die Eingangsbildfunktion wiederholt mit dem Kern der Gaußschen Funktion gefaltet und wiederholt unterabgetastet wird.Diese Methode wird hauptsächlich für die Implementierung des SIFT Algorithmus verwendet. Jede Bildschicht hängt jedoch von der vorherigen Bildschicht ab, und das Bild muss in der Größe angepasst werden.Daher hat diese Berechnungsmethode eine große Kosten in Berechnung. Dagegen in SURF Algorithmus, die Verfahre ist durch die Erhöhung der Größe des Bildkerns. Diese ist auch der Unterschied zwischen dem SIFT Algorithmus und dem SURF Algorithmus bei der Verwendung des Pyramidenprinzips.
Der Algorithmus ermöglicht, dass mehrere Bilder des Maßstabsraums gleichzeitig verarbeitet werden, ohne dass das Bild unterabgetastet wird, wodurch die Leistung des Algorithmus verbessert wird. Das linke Bild in Abbildung 4.2 ist eine Pyramidenstruktur, die auf herkömmliche Weise erstellt wird, die Größe des Bildes wird geändert, und die Operation wird die Unterebene  unter Verwendung der Gaußschen Funktion wiederholt glätten. Der Surf Algorithmus auf der rechten Seite in Abbildung 4.2 behält das ursprüngliche Bild unverändert und ändert nur die Filtergröße.

\begin{figure}[htb]
 \centering 
 \includegraphics[keepaspectratio,width=0.8\textwidth]{images/4_ZweiteErfahrung/Scale_space.pdf}
 \caption{Scale space}
 \label{fig:Scale space}
\end{figure} 


$\bullet$ \textbf{Präzise Lokalisierung von Feature-Punkten}\\
Vergleichen die Größe jedes Pixel, das von der hessischen Matrix verarbeitet wird, mit 26 Punkten in seiner drei Dimensionen Raum, wie in Abbildung 4.3 zeight. Wenn es das Maximum oder Minimum dieser 26 Punkte ist, wird es als vorläufiger Merkmalspunkt beibehalten. Das dreidimensionale lineare Interpolationsverfahren wird verwendet, um die Merkmalspunkte des Subpixel-Niveaus zu erhalten, und die Punkte, deren Werte kleiner als ein bestimmter Schwellenwert sind, werden ebenfalls entfernt.

\begin{figure}[htb]
 \centering 
 \includegraphics[keepaspectratio,width=0.4\textwidth]{images/4_ZweiteErfahrung/Extreme_Wert_Erkennung.pdf}
 \caption{Extreme Wert Erkennung}
 \label{fig:Extreme Wert Erkennung}
\end{figure} 


$\bullet$ \textbf{Hauptrichtungsermittlung}\\
SIFT wählt die Hauptrichtung des Merkmalspunkts unter Verwendung des Gradientenhistogramms im Merkmalspunktfeld aus. Die Richtung, in der der Bin-Wert des Histogramms der größte und oder 80\% maximale Bin-Wert  überschreitet, wird als Hauptrichtung des Merkmalspunkts genommen. Dagegen beim SURF wird das Gradientenhistogramm nicht statistiken, sondern das Harr-Wavelet-Eigenshcaft im Merkmalspunkt-Bereich wird statistisch analysiert. Das heißt, im Bereich der Merkmalspunkt (zum Beispiel innerhalb eines Kreises mit einem Radius von 6s, wobei s der Maßstab ist, auf dem der Punkt liegt) die Summe der Horizontal-Haar-Wavelet-Merkmale und der Vertikal-Haar-Wavelet-Merkmale aller Punkte im  60-Grad-Sektor($\pi/3$) werden gezählt. Die Größe des Haar Wavelets stellt als 4s, so dass für jeden Sektor einen Wert bekommt. Dann wird 60-Grad-Sektor in einem bestimmten Intervall gedreht, schließlich lassen die Richtung des Sektors mit Maximalwert als Hauptrichtung des Merkmalspunkts nehmen. Ein schematisches Diagramm des Prozesses ist wie folgt in Abbildung 4.4.

\begin{figure}[htb]
 \centering 
 \includegraphics[keepaspectratio,width=0.8\textwidth]{images/4_ZweiteErfahrung/Dominante_Orientierung_Feststellen.pdf}
 \caption{Dominante Orientierung Feststellen}
 \label{fig:Dominante Orientierung Feststellen}
\end{figure} 


$\bullet$ \textbf{Merkmalspunkt Deskriptor Generierung}\\
SURF nehmen eine quadratische Rahmen um den Merkmalspunkt. Die Seite der Rahmen ist 20s (s ist die Skala, bei der der Merkmalspunkt erkannt wird). Die Richtung des Rahmens ist natürlich die Hauptrichtung, die in vorliegened Schritt erfasst wird. Die Rahmen wird dann in 16 Unterbereiche unterteilt, von denen jeder die Haar-Wavelet-Merkmale der horizontalen und vertikalen Richtungen von 25 Pixeln berechnen. Hier die horizontalen und vertikalen Richtungen sind relativ zur Hauptrichtung. Das Haar Wavelet-Merkmal ist die Summe der horizontalen Richtungswerte, die Summe der absoluten Werte in der horizontalen Richtung, die Summe der vertikalen Richtungen und die Summe der absoluten Werte in der vertikalen Richtung. Das schematisches Diagramm in Abbildung 4.5 zeigt dieses Prozesse.

\begin{figure}[htb]
 \centering 
 \includegraphics[keepaspectratio,width=0.5\textwidth]{images/4_ZweiteErfahrung/Merkmalspunkt_Deskriptor.pdf}
 \caption{Merkmalspunkt Deskriptor}
 \label{fig:Merkmalspunkt Deskriptor}
\end{figure} 

Auf diese Weise hat jeder kleine Bereich 4 Werte, so dass jeder Merkmalspunkt ein $16*4=64$ dimensionaler Vektor verfügt, der halb so klein wie Sift ist, deswegen den Anpassungsprozess beim Merkmalanpassungsprozess stark beschleunigt. Die folgende Abbildung 4.6 zeigt den Merkmalspunkt, den wir durch den SURF-Algorithmus erhalten haben.

\begin{figure}[htb]
 \centering 
 \includegraphics[keepaspectratio,width=1.0\textwidth]{images/4_ZweiteErfahrung/SURF_Detektion.pdf}
 \caption{SURF Detektion}
 \label{fig:SURF Detektion}
\end{figure} 


\subsection{RANSAC}
Wie in Abbildung 4.6 gezeigt,  nur durch SURF Algorithmus gibt es immer einige offensichtliche Mismatch-Punkte. Zur Lösung des Problems ist es immer erforderlich mit \gls{ransac}.





\begin{figure}[htb]
 \centering 
 \includegraphics[keepaspectratio,width=0.8\textwidth]{images/4_ZweiteErfahrung/RANSAC/OhneRANSAC.pdf}
 \caption{OhneRANSAC}
 \label{fig:OhneRANSAC}
\end{figure} 

\begin{figure}[htb]
 \centering 
 \includegraphics[keepaspectratio,width=0.8\textwidth]{images/4_ZweiteErfahrung/RANSAC/MitRANSAC.pdf}
 \caption{MitRANSAC}
 \label{fig:MitRANSAC}
\end{figure} 

\subsection{Kamara Model}

\subsection{Parameter Optimierung}

\section{Differenzbild}


\section{Image Processing} 


\section{QR Pattern Detection} 


