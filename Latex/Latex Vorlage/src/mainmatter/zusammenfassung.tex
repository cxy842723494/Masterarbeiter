\chapter{Zusammenfassung} \label{cha:Zusammenfassung}

In dieser Arbeit werden zwei Verfahren zur Ausschnittsdetektion für die Screen-Camera Visible Light Communication untersucht und implementiert. Methode 1 ist vom differentiellen Modulationsverfahren des David-Systems inspiriert. d.h. Daten werden mit geringer Amplitude im Videosignal differentiell überlagert. In der Methode wird ein QR Muster an jeder Ecke der Datenebene hinzugefügt und dann mit den Daten zusammen hinter dem Bild moduliert. Diese Methode kann nicht nur den unschönen Effekt lösen, welcher durch direktes Hinzufügen des QR Musters zu dem Videobild verursacht wird, sondern kann auch den QR Muster durch seine spezifische geometrische Eigenschaft, d.h. in jeder Richtung beträgt das Breiteverhältnis $1:1:3:1:1$, effektiv erkennen. Dadurch wird schließlich der Modulationsbereich bestimmt. Das 2. Verfahren wird in den physikalischen Eigenschaften des Modulationsbereichs ausgeführt. Im \gls{david} System wirkt der Bildschirm als Übertragungsende, und der Bildschirmbereich ist der Modulationsbereich, der im Allgemeinen in Form eines Rechtecks ist. Dadurch kann dieses Problem in eine rechteckige Erkennung umgewandelt werden. Durch die Radon Transformation wird die längste gerade Linie in dem Bild detektiert, um ein Rechteck bzw. den Modulationsbereich, zu bestimmen.

zwei Aufnahmebedingungen wurden in dieser Arbeit berücksichtigt, eine ist die Aufnahme mit dem Smartphone aus der Hand und die andere auf einem Stativ. Handheld-Kamera-Shooting ist eine der häufigsten und bequemsten Aufnahmemethoden im Alltag, und hat eine unerlässliche Position in der praktischen Anwendung. Zu diesem Zweck wurde ein Bildregistrierungsmodul speziell für Handschütteln entwickelt, welche Merkmals Detektion, \gls{ransac}, Kamera Modell und Nichtlineare Optimierung beinhaltet. Beim Testen beträgt die relative Verschiebung zwischen den entsprechenden Punkten des korrigierten Bildes c.a. 0.3 Pixel. Dies führt zu einem relativ guten Differenzbild. Dieser spielte eine wichtige Rolle weil beide Verfahren in dieser Arbeit auf der Detektion von Differenzbildern basieren. Dafür wird der Begriff $ ``Energie" $ eingeführt, welcher die Klarheit des Modulationsbereichs repräsentiert. Gemäß der Energiesortierung werden die klarsten Bilder ausgewählt und hinzugefügt, um das zu detektierende Bild zu erhalten. In Methode 1 werden die beide obigen Schritte implementiert, und durch Auswertung die durchschnittliche Abweichung ca. $ 1-2 $ Pixel bemessen. Es sollte hier angemerkt werden, dass aufgrund der Eigenschaften der Bildregistrierung das Video, das in Methode 1 aufgenommen wurde, ein statisches Video ist, d.h. der Inhalt des Videos hat sich nicht geändert. Der Grund dafür ist, dass der Punkt der Merkmals Detektion hauptsächlich auf dem Inhalt des Videos basiert. Wenn sich der Videoinhalt stark ändert, ist es schwierig, die Genauigkeit der Bildwiederherstellung zu garantieren. Einige bei der Arbeit versuchten Methoden war die Abschirmung des zentralen Bereichs während der Merkmals Detektion, d.h. nur die Merkmale in der Umgebung des Bildes wurden erkannt, was nicht zufriedenstellend ist. Methode 2 implementiert die Erkennung des Modulationsbereichs unter Verwendung eines Stativs. Nach der Erkennung beträgt die durchschnittliche Abweichung ebenfalls ca. $ 1-2 $ Pixel. In beiden Methoden kann der Modulationsbereich mit Parametern, wie die Modulationsapmlitude als 4,  der Datenblock als $ 4 \times 4 Pixel$, erfolgreich bestimmt werden. 

Für dieses Thema gibt es einige mögliche weitere Forschung in der Zukunft. Zuerst kann wie oben erwähnt, ein Verfahren gefunden werden, welches beim Aufnehmen des dynamischen Videos mit dem Smartphone aus der Hand, das Handschütteln-Problem effektiv löst. Der Fokus liegt darauf, die Wirkung vom Video-Inhalt zu neutralisieren und das Bild nur auf die Merkmale des Hintergrunds zu filtern. Zweitens, die in dieser Arbeit verwendete Differenzbild Optimierungsmethode, kann aufgrund der Berechnung der $ ``Energie" $ für jedes Differenzbild zu einer großen Rechenlast führen. Ein effektives Verfahren wäre versuchen, ein zu detektierendes Bild, das auf dem bekannten Differenzbild basiert, effizient zu generieren. Schließlich wurde in Experiment festgestellt, dass unter denselben Parametern die Verwendung des gleichen Smartphones mit dem gleichen Video-Inhalt auf verschiedenen Bildschirmen, zu unterschiedlichen Ergebnissen führt. Dadurch können die Auswirkungen der Bildschirmfaktoren, wie z.B. Beleuchtungsart, Auflösung, Bildwiederholfrequenz, weiter erforscht werden.




