\chapter{Zusammenfassung} \label{cha:Zusammenfassung}

In dieser Arbeit werden zwei Verfahren zur Ausschbittsdetektion für die Screen-Camera Visible Light Communication untersucht und implementiret. Die erste Methode ist vom differentiellen Modulationsverfahren des David-Systems inspiriert. d.h. Daten werden mit geringer Amplituden dem Videosignal differentiell überlagert. In der Methode wird ein QR Muster an jeder Ecke der Datenebene hinzugefügt und dann mit den Daten zusammen hinter dem Bild moduliert. Diese Methode kann nicht nur das unschöne Effekt lösen, das durch direktes Hinzufügen des QR Musters zu dem Videobild verursacht wird, sondern kann auch den QR-Muster durch dessen spezifische geometrische Eigenschaft, d.h. in jeder Richtung ist das Breiteverhältnis $1:1:3:1:1$ beträgt, effektiv erkennen. Dadurch wird schließlich der Modulationsbereich bestimmt. Das zweite Verfahren wird in den physikalischen Eigenschaften des Modulationsbereichs ausgeführt. In \gls{david} System wirkt der Bildschirm als Übertragungsende, und der Modulationsbereich ist natürlich der Bildschirmbereich, der allgemein in Form eines Rechtecks existiert. Dann dieses Problem kann in eine rechteckige Erkennung umgewandelt werden. Durch die Radon Transformation wird die längste gerade Linie in dem Bild detektiert, um die Rechteck bzw. Modulationsbereich, zu bestimmen.

zwei Aufnahmebedingungen wurden in dieser Arbeit berücksichtigt, einer ist für die Aufnahme die Smartphone in der Handy gehalten und der andere ist dafür auf einem Stativ. Handheld-Kamera-Shooting ist eine der häufigsten und bequemsten Aufnahmemethoden im täglichen Leben, und es hat eine wichtige Bedeutung in der praktischen Anwendung. Zu diesem Zweck wurde ein Bildregistration Modul speziell für Handshake entwickelt, welche beinhaltet Featur Detektion, \gls{ransac}, Kamera Model und Nichtlineare Optimierung. Dadurch beim Testen beträgt der relative Verschiebung zwischen den entsprechenden Punkten des korrigierten Bildes c.a. 0.3 Pixel. Dies führt zu einem relativ gut Differenzbild. Er spielte eine wichtige Rolle, while die beide Verfahren in diesem Arbeit auf der Detektion von Differenzbild basieren. Dafür wird ein Begriff $ ``Energie" $ eingeführt, welche die Klarheit des Modulationsbereichs repräsentiert. Gemäß der Energiesortierung werden die paar klarsten Bilder ausgewählt und hinzugefügt, um das zu detektierend Bild zu erhalten. In Methode 1 wird das obige beide Schritt implementiert, und durch Auswertung wird der durchschnittliche Offset beträgt ca. $ 1-2 $ Pixel. Es sollte hier angemerkt werden, dass aufgrund der Eigenschaften der Bildregistrierung das Video, das in 1.Methode aufnommen wird, statische Videos sind, dh der Inhalt des Videos hat sich nicht geändert. Der Grund dafür ist, dass der Punkt der Merkmale-Erkennung hauptsächlich auf dem Inhalt des Videos basiert. Wenn sich der Videoinhalt stark ändert, ist es schwierig, die Genauigkeit der Bildwiederherstellung zu garantieren. Einige Methoden wurden bei der Arbeit versucht, wie die Abschirmung des zentralen Bereichs während des Merkmale Detektion, dh. nur die Merkmale in der Umgebung des Bildes erkannt werden, aber der Effekt ist nicht zufriedenstellend. Die 2. Methode implementiert die Erkennung des Modulationsbereichs mit Verwendung eines Stativs. Nach die Erkennung wird der durchschnittliche Offset beträgt auch ca. $ 1-2 $ Pixel. Beiden Methode kann die Modulationsbereich mit Parameter, wie Modulationsapmlitude als 4, Datenblock als $ 4 \times 4 Pixel$, erfolgreich bestimmen. 

Für diese Thema gibt es einige mögliche weiter Erforschen in der Zukunft. Zuerst wie oben erwähnt, kann ein Verfahren suchen, imdem bei Aufnehmen des dynamischen Videos die Smartphone in der Handy gehalten, um HandyShake-Problem effektiv zu lösen. Der Fokus liegt darauf, die Auswirkungen von Video-Inhalt zu blockieren und das Bild nur auf der Merkmale des Hintergrunds wiederherzustellen. Zweitens, die in dieser Arbeit verwendet Differenzbild Optimierungsmethode, aufground der Berechnung des $ ``Energie" $s für jedes Differenzbild zu einer großen Rechenlast führen. Eine effektiv Verfahren kann versuchen, um ein zu detektierendes Bild, das auf dem bekannten Differenzbild basierend, effizient zu entstehen. Schließlich wurde in dem Experiment festgestellt, dass unter denselben Parametern die Verwendung des gleichen Smartphone zum dem gleichen Video-Inhalt auf verschiedenen Bildschirmen, zu ein unterschiedliche Ergebnis führen würde. Dann kann die Auswirkungen der Bildschirmfaktoren, wie z.B. Beleuchtungsart, Auflösung, Bildwiederholfrequenz, ein weiter Erforschen durchführen.




