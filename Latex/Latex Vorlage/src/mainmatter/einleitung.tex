\chapter{Einleitung} \label{cha:Einleitung}

In den letzten Jahren hat \gls{vlc} sowohl in der Wissenschaft als auch in der Industrie große Aufmerksamkeit erregt. Aufgrund seiner Vorteile hinsichtlich Bandbreitenverfügbarkeit, Sicherheit und Datensicherheit, \gls{vlc} wird in vielen Anwendungen als eine bessere Alternative zu Radiofrequenz und Infrarot betrachtet, wie z.B. drahtloses Netzwerk mit Beleuchtungssystem\cite{1205458}, Innenraumpositionierung \cite{4649677}, optische Verbindungen zu elektronischen Chips \cite{867694}, Körpersensornetzwerken \cite{bodysensor}, usw. Viele dieser \gls{vlc}-Techniken wurden entwickelt, um vorhandene Lichterzeugungsgeräte zu nutzen. Ein innovativer Ansatz ist die Verwendung von Display-Kamera-Paaren zur Datenübertragung. Aufgrund der steigenden Performance der beteiligten Schlüsselkomponenten erscheinen Datenraten von bis zu 100 Mbit/s realisierbar, während gleichzeitig eine Videopräsentation für menschliche Zuschauer auf dem gleichen Bildschirm zur Verfügung gestellt werden kann. Mit einem solchen System können viele innovative Medienanwendungen realisiert werden.

\section{Motivation} 

In einer aktuellen Forschung untersucht der Lehrstuhl für Kommunikationstechnik an der TU Dortmund ein neuartiges Verfahren der Visible Light Communication, d.h. \gls{david}-System, indem eine optische Freiraum-Übertragung unter Verwendung verfügbarer Anzeige und Kamera durchführen. Ein Highlight dieses Systems ist, dass sie die ursprüngliche Funktionalität des Displays nicht beeinträchtigt; Während der Datenübertragung kann das Display immer noch statische Bild oder Videos anzeigen, ohne dass menschliche Betrachter die versteckten Datensignale wahrnehmen. Die Übertragungsdaten werden einem Videosignal in Form geringer Amplitudenänderungen differentiell überlagert. Am Empfängen lassen sich der Display durch ein Kamera oder Smartphone aufnehmen. Anschließend können die überlagerte Daten empfangen und decodiert werden. Hierzu ist es notwendig, den Modulationsbereich, der durch die optische Projektion verzerrt wird, zu detektieren.

\section{Aufgabe in dieser Arbeit} 

In dieser Arbeit wird zwei Verfahren zur Ausschnittsdetektion für Differenzbilder untersucht und implementiert. Die erste Methode verwendet der Charakteristiken der Datenmodulation des \gls{david} Systems, d.h. auf Differenzbild wird das QR Muster detektiert, um das Modulationsbereich zu bestimmen. Die zweite Verfahren basiert auf der Geometrie des Displays. Mit Verwendung der Radon Transformation kann das rechteckig Modulationsbereich bestimmt werden. Darüber hinaus wird in dieser Arbeit eine Bildregistration Modul entwickelt, um die Einfluss, bei Aufnahme des Videos die Smartphone in der Handy gehalten, auszugleichen. Außerdem durch Einführung der Begriff $ ``Energie" $ wurde eine Optimierungsmethode für Differenzbild entwickelt. In Anschluss wurden die Performance und die Ersetzbarkeit der beiden Verfahren evaluiert. Der zweite Verfahren wurde auf einem Smartphone-GPU implementiert.

\section{Aufbau der Arbeit} 

Der Rest des Dokuments ist wie folgt organisiert.

Das nachfolgende Kapitel gibt einen Beschreibung über \gls{david} System. Es stellt die Struktur und Arbeitsweise des \gls{david} Systems vor und listet verschiedene Anwendungsbereiche auf, die möglicherweise vom \gls{david}-Konzept profitieren könnten.

Der dritte und vierte Absatz enthält Informationen zur beiden Verfahren. Die Struktur und Zusammensetzung der Methode werden vorgestellt und das Arbeitsprinzip jedes Teils wird detailliert beschreibt.

Das nächste fünfte Kapitel zeigt die Implementierung jeder Methode. Darin können die Effekt jedes Teils in der Methode sehen. %Schließlich zeigt die Implementierung auf eine Smartphone-GPU.

Kapitel 6 enthält eine Evaluierung der beiden Verfahren. Die Performance und die Einsetzbarkeit des Verfahrens werden analysiert.

Der letzte Abschnitt schließt die Ergebnisse ab und gibt einen Ausblick auf die zukünftige Arbeit.