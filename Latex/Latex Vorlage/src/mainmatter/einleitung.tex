\chapter{Einleitung} \label{cha:Einleitung}

In den letzten Jahren hat \gls{vlc} sowohl in der Wissenschaft als auch in der Industrie große Aufmerksamkeit erregt. Aufgrund seiner Vorteile hinsichtlich Bandbreitenverfügbarkeit, Sicherheit und Datensicherheit wird \gls{vlc} in vielen Anwendungen als eine bessere Alternative zur Radiofrequenz und Infrarot betrachtet, wie z.B. drahtloses Netzwerk mit Beleuchtungssystem\cite{1205458}, Innenraumpositionierung \cite{4649677}, optische Verbindungen zu elektronischen Chips \cite{867694}, Körpersensornetzwerken \cite{bodysensor}, usw. Viele dieser \gls{vlc}-Techniken wurden entwickelt, um bereits vorhandene Lichterzeugungsgeräte zu nutzen. Ein innovativer Ansatz ist die Verwendung von Display-Kamera-Paaren zur Datenübertragung. Aufgrund der steigenden Performance der beteiligten Schlüsselkomponenten erscheinen Datenraten von bis zu 100 Mbit/s realisierbar, während gleichzeitig eine Videopräsentation für menschliche Zuschauer auf dem gleichen Bildschirm zur Verfügung gestellt werden kann. Mit einem solchen System können viele innovative Medienanwendungen realisiert werden.

\section{Motivation} 

In einer aktuellen Forschung untersucht der Lehrstuhl für Kommunikationstechnik an der TU Dortmund ein neuartiges Verfahren der Visible Light Communication: Das \gls{david}-System, welches eine optische Freiraum-Übertragung unter Verwendung verfügbarer Displays und Kameras durchführt. Ein Highlight dieses Systems ist, dass sie die ursprüngliche Funktionalität des Displays nicht beeinträchtigt; Während der Datenübertragung kann das Display immer noch statische Bilder oder Videos anzeigen, ohne dass menschliche Betrachter die versteckten Datensignale wahrnehmen. Die Übertragungsdaten werden in einem Videosignal in Form geringer Amplitudenänderungen differentiell überlagert. Am Empfänger lässt sich der Display durch eine Kamera oder Smartphone optische aufnehmen. Anschließend können die überlagerte Daten empfangen und decodiert werden. Hierzu ist es notwendig, den Modulationsbereich, der durch die optische Projektion verzerrt wird, mit einigen Operationen wiederherzustellen.

In dieser Arbeit werden zwei Verfahren zur Ausschnittsdetektion für Differenzbilder untersucht und implementiert. Die erste Methode verwendet die Charakteristiken der Datenmodulation des \gls{david} Systems, d.h. auf dem Differenzbild wird das QR Muster detektiert, um den Modulationsbereich zu bestimmen. Das zweite Verfahren basiert auf der Geometrie des Displays. Mit Verwendung der Radon Transformation kann der rechteckige Modulationsbereich bestimmt werden. Darüber hinaus wird in dieser Arbeit ein Modul zu Bilderkennung entwickelt, welches die Videostabilität bei Aufnahme mit dem Smartphone aus der Hand verbessern kann. Außerdem durch Einführung des Begriffs der $ ``Energie" $ wurde eine Optimierungsmethode für Differenzbilder entwickelt. Im Anschluss wurde die Performance der beiden Verfahren evaluiert. Das zweite Verfahren wurde auf einer Smartphone-GPU implementiert.

\section{Aufbau der Arbeit} 

Der Rest des Dokuments ist wie folgt organisiert. Das nachfolgende Kapitel gibt eine Beschreibung über das \gls{david} System. Es stellt die Struktur und Arbeitsweise des \gls{david} Systems vor und listet verschiedene Anwendungsbereiche auf, die möglicherweise vom \gls{david}-Konzept profitieren könnten. Der dritte und vierte Absatz enthält Informationen zu beiden Verfahren. Die Struktur und Zusammensetzung der Methoden werden vorgestellt und das Arbeitsprinzip jedes Teils wird detailliert beschreibt. Das nächste, also fünfte Kapitel, zeigt die Implementierung jeder Methode. Darin können die Wirkung jedes Teils in der Methode gesehen werden. %Schließlich zeigt die Implementierung auf eine Smartphone-GPU.
Kapitel 6 enthält eine Evaluierung der beiden Verfahren. Die Performance des Verfahrens werden analysiert. Der letzte Abschnitt schließt die Ergebnisse ab und gibt einen Ausblick auf die zukünftige Arbeit.