\RequirePackage[l2tabu, orthodox]{nag}		%checks for obsolete LaTeX packages and outdated commands
\documentclass[
	fontsize=11pt,%				Schriftgroesse 11pt
	paper=a4,%					Layout fuer Din A4
	twoside=false,%				Layout fuer beidseitigen Druck %semi %true
	twocolumn=false,%			zweispaltiger Satz
	headsepline=true,%			horizontale Linie unter Kolumnentitel
	footsepline=false,%			Trennline zum Seitenfuß
	headinclude=true,%			Kopfzeile wird Seiten-Layouts mit beruecksichtigt
	footinclude=false,%			Fußzeile wird Seiten-Layouts mit beruecksichtigt
	chapterentrydots=false,%	Inhaltsverzeichnis dotfill für Kapitel
	parskip=false,%				Der erste Absatz eines Abschnitts wird nicht eingezogen
	BCOR=12mm,%					Korrektur fuer die Bindung
	DIV=15,%					DIV-Wert fuer die Erstellung des Satzspiegels, siehe scrguide	%calc
	parskip=half,%				Absatzabstand statt Absatzeinzug
	footnotes=multiple,%		Fußnotenmarkierungen %nomultiple
	headings=openany,%			Kapitel können auf geraden und ungeraden Seiten beginnen %openleft %openright
	index=default,%				Index im Inhaltsverzeichnis %totoc %numbered
	bibliography=totoc,%		Literaturverz. Wird ins Inhaltsverzeichnis eingetragen
	listof=totoc,%				Tabellen und Abbildungs. Wird ins Inhaltsverzeichnis eingetragen
	numbers=noendperiod,%		Kapitelnummern immer ohne Punkt	%endperiod %autoendperiod
	draft=false,%				Keine Bilder in der Anzeige, overfull hboxes werden angezeigt
	titlepage=true,%			Titlei auf erster Seite	%firstiscover
	chapterprefix=false,%		Kapitelprefix für mainmatter
	appendixprefix=true,%		Kapitelprefix für Anhang
	bibliography=oldstyle,%		Formatierungsstil des Literaturverzeichnisses %openstyle
	cleardoublepage=plain,%		Leere seiten plain
	pagesize=auto,%				Ausgabetreiber %pdftex
	]{scrbook}%					crbook 2018/01/23 v3.24

\usepackage{scrhack}
\usepackage{scrwfile}
\usepackage{ifluatex}

\usepackage[english, ngerman]{babel}%		Nötig, damit Dokument in Deutsch generiert wird
\usepackage[utf8]{luainputenc}%				Nötig um Umlaute zu tippen. Alternativ: latin1
\ifluatex
	\usepackage[no-math]{fontspec}
\else
	\usepackage[T1]{fontenc}%				font encoding
\fi

\usepackage[centertags]{amsmath}%	AMS-Mathematik, %centertags zentriert Nummer bei split, %fleqn
\interdisplaylinepenalty=2500
\usepackage{amsthm}
\usepackage{amsfonts}%				Standard fonts
\usepackage{amssymb}%				Standard fonts
\usepackage{gensymb}
\usepackage{array}
\usepackage{mathtools}%				Multiline and split equations
\usepackage{relsize}
\usepackage{physics}

\ifluatex
	\usepackage[luatex]{graphicx}%			Package um grafiken zu includieren
\else
	\usepackage[pdftex]{graphicx}%
\fi
\usepackage{textcomp}%						verschiedene Symbole
\usepackage[onehalfspacing]{setspace}%		Zeilenabstand anpassen
	\KOMAoptions{DIV=last}%					Seitenspiegel durch setspace nicht verändern
\usepackage[hang, bf, font=small, textformat=period]{caption}%	Captions für figures und tables
\usepackage{subcaption}%					Package um subfigure zu erstellen
\usepackage[svgnames, table]{xcolor}%		Farbkonversion etc.
\usepackage{colortbl}%						Support for colors in tables
\usepackage{multirow}%						Zelle einer Tabelle über mehrere Zeilen ausdehnen
\usepackage{booktabs}%						Tabellenformatierung
\usepackage{rotating}%						Rotation von tables und figures
\usepackage{tikz}%							Zeichnen direkt in Latex
\usepackage[siunitx]{circuitikz}%			Zeichnen von elektrischen Schaltbildern direkt in Latex
\usepackage{pgfplots}
	\pgfplotsset{
		compat=newest,%						% TODO: set to current version of pgfplots, when starting your work (2018/01/23: compat=1.15)
		width=0.8\textwidth,%
		colormap={blackwhite}{gray(0cm)=(0); gray(1cm)=(1)},%
	}
\usepackage{siunitx}
	\sisetup{
		locale = DE,%
		detect-all,%
% 		scientific-notation = engineering,%
		list-final-separator = { und },%
		list-pair-separator = { und },%
		range-phrase = { bis },%
		per-mode = symbol,%
		binary-units = true,%
		quotient-mode = fraction,%
		output-complex-root = j,%
	}
	\DeclareSIUnit[]\dBm{dBm}
	\DeclareSIUnit[]\dBW{dBW}
\usepackage{calc}%							Hilfspaket um mit Latex variablen zu rechnen
\usepackage{listings}%						Paket um sourcecode zu publizieren

\lstnewenvironment{matlab}[1][]{
	\lstset{
		backgroundcolor=\color{white}, %\color{TUgreen!20}, %\color{gray!20}
		fillcolor=,% %\color{black},
		captionpos=t,% b
		tabsize=4,%
		rulecolor=\color{black},%
		rulesepcolor=\color{gray!30},%
		language=Matlab,%
		basicstyle=\small,%
		numbers=left,%
		numberfirstline=true,%
		numberblanklines=true,%
		stepnumber=1,%
		numbersep=5pt,%
		numberstyle=\scriptsize,%
		upquote=true,%
		aboveskip={0\baselineskip},%
% 		belowskip=\baselineskip,%
		columns=flexible,% %fixed,
		showstringspaces=false,%
		extendedchars=true,%
		breaklines=true,%
		breakatwhitespace=true,%
		prebreak = \raisebox{0ex}[0ex][0ex]{\ensuremath{\color{black}\hookleftarrow}},%
		breakautoindent=true,%
		frame=single,% %tblr, %shadowbox, %tblrTBLR,
		frameround=ffff,% %tttt,
		showtabs=false,%
		showspaces=false,%
		showstringspaces=false,%
		identifierstyle=\ttfamily,%
		basicstyle=\ttfamily\color{black},%
		keywordstyle=\ttfamily\bfseries\color{MidnightBlue},%
		commentstyle=\color{tudark},% %\color{MediumTurquoise},
		stringstyle=\color{DarkOrchid},%
		mathescape=true,%
		showlines=true,%
		escapechar=§,%							escape to latex
		abovecaptionskip=0\smallskipamount,%
		belowcaptionskip=3\smallskipamount,%
		xleftmargin=1.25\baselineskip,%
		xrightmargin=0pt,
		#1,%									add more options from the optional parameter
	}}{}
\usepackage[
	pdfpagelabels,
	colorlinks=false,
	plainpages=false,
	pdfborder={0 0 0},
	]{hyperref}%						Automatische Kreuzreferenzierungen (auch URLs) im PDF
\makeatletter
\AtBeginDocument{\hypersetup{
		pdftitle = {\@documentnumber},%
		pdfsubject = {\@title},%
		pdfauthor = {\@author},%
		pdfkeywords = {},%
		bookmarksopen = {true},%
		bookmarksdepth = 3,%			up to subsubsection titles will be included into PDF structure
		bookmarksopenlevel = 1,%		... up to section titles
	}
}
\usepackage[acronym,nonumberlist,section=chapter,translate=false,shortcuts,toc=true]{glossaries} % Glossar, Abkürzungsverzeichniss etc.
	\renewcommand{\acronymname}{Abkürzungsverzeichnis}
	\newglossary[slg]{symbol}{sym}{sbl}{Symbolverzeichnis}
	\makeglossaries
	\newglossarystyle{tab_style}{
		\renewenvironment{theglossary}{\begin{longtable}[l]{llp{0mm}}}{\end{longtable}}
		\renewcommand*{\glsgroupheading}[1]{}
		\renewcommand*{\glossentry}[2]{\glsentryitem{##1}\glstarget{##1}{\textbf{\glossentryname{##1}}} & \glossentrydesc{##1} & ##2 \tabularnewline[3px]}
		\renewcommand*{\subglossentry}[3]{\glossentry{##2}{##3}}
		\renewcommand*{\glsgroupskip}{}
	}
	\newglossarystyle{tab_style_sym}{ 
		\renewenvironment{theglossary}{\begin{longtable}[l]{llp{0mm}}}{\end{longtable}}
		\renewcommand*{\glsgroupheading}[1]{}
		\renewcommand*{\glossentry}[2]{\glsentryitem{##1}\glstarget{##1}{\glossentrysymbol{##1}} & \glossentrydesc{##1} & ##2 \tabularnewline[3px]}
		\renewcommand*{\subglossentry}[3]{\glossentry{##2}{##3}}
		\renewcommand*{\glsgroupskip}{}
	}

%%% Symbols
%
\newglossaryentry{pi}{type=symbol,name=\ensuremath{\pi},symbol={\ensuremath{\pi}},sort=pi,description={ratio of circumference of circle to its diameter}}
\newglossaryentry{ohm}{type=symbol,name=ohm,symbol={\ensuremath{\Omega}},description=unit of electrical resistance}
\newglossaryentry{angstrom}{type=symbol,name={\aa}ngström,symbol={\AA},sort=angstrom,description={non-SI unit of length}}
\newglossaryentry{angstro}{type=symbol,name=angstrom,symbol={A},description=non-SI unit of length}


%%% Acronyms
\newacronym{led}{LED}{\textbf{l}ight-\textbf{e}mitting \textbf{d}iode}
\newacronym{ieee}{IEEE}{Institute of Electrical and Electronics Engineers}
\newacronym{abc}{abc}{test}
\newacronym{eeprom}{EEPROM}{electrically erasable programmable read-only memory}%	load symbol and acronym defintions
\glsaddallunused[\acronymtype]%				add all abbreviations
\glsaddallunused[symbol]%					add all symbols
\makeatother

% *** CITATION PACKAGES ***
\usepackage[style=ieee, backref=false, dashed=false, doi=false, isbn=false, backend=biber]{biblatex}
\usepackage[autostyle=true, german=quotes]{csquotes}
\addbibresource{literatur.bib}

\DeclareBibliographyDriver{standard}{%
	\usebibmacro{bibindex}%
	\usebibmacro{begentry}%
	\usebibmacro{author}%
	\setunit{\labelnamepunct}\newblock
	\usebibmacro{title}%
	\newunit\newblock
	\printfield{number}%
	\setunit{\addspace}\newblock
	\printfield[parens]{type}%
	\newunit\newblock
	\usebibmacro{location+date}%
	\newunit\newblock
	\iftoggle{bbx:url}
		{\usebibmacro{url+urldate}}
		{}%
	\newunit\newblock
	\usebibmacro{addendum+pubstate}%
	\setunit{\bibpagerefpunct}\newblock
	\usebibmacro{pageref}%
	\newunit\newblock
	\usebibmacro{related}%
	\usebibmacro{finentry}}

\usepackage[german]{cleveref}

% WASSERZEICHEN
% \usepackage{draftwatermark}
% \SetWatermarkText{Draft}
% \SetWatermarkLightness{0.9}
% \SetWatermarkScale{1}

% Schrift
\ifluatex
	\setmainfont{Arial}
	\setsansfont{Arial}
	\usepackage[italic]{mathastext}
\else
	\usepackage{helvet}
	\renewcommand{\familydefault}{\sfdefault}
	\usepackage[helvet]{sfmath}
	\usepackage{sansmathaccent}
\fi

% TU Farben
\xdefinecolor{tugreen}{RGB}{132, 184, 24}%		0
\colorlet{tulight}{tugreen!20!white}%			1
\colorlet{tudark}{tugreen!60!black}%			2
\xdefinecolor{tuorange}{RGB}{227, 105, 19}%		3
\xdefinecolor{tuyellow}{RGB}{242, 189, 0}%		4
\xdefinecolor{tucitron}{RGB}{249, 219, 0}%		5

% Nummerierung in den Überschriften bis Ebene...
% \setcounter{secnumdepth}{4}
% \setcounter{tocdepth}{4}

% Fußnoten mit Buchstaben
% \renewcommand{\thefootnote}{\alph{footnote}}

% Verzeichnisnamen
% \renewcommand{\contentsname}{}
% \renewcommand{\listfigurename}{}
% \renewcommand{\listtablename}{}
% \renewcommand{\refname}{}
% \renewcommand{\abstractname}{}
\renewcommand*{\lstlistingname}{Quellcode} % Listings umbenennen
\renewcommand{\lstlistlistingname}{Quellcodeverzeichnis} % Listingsverzeichnis umbenennen

% Todo environment
\newenvironment{TODO}[1]{\textcolor{red}{\textbf{TODO: #1}}}

% Bilder
\graphicspath{./images/}
\DeclareGraphicsExtensions{.pdf,.png,.jpg}

% Titelei
\usepackage{kt_title}
\usepackage{lipsum}	% TODO: remove. This is for dummy text only

%%%%%%%%%%%%%%%%%%%%%%%%%%%%%%%%%%%%%%%%%%%%%%%%%%%%%%%%%%%%%%%

\thesistitle{Tasty Kanalmodell\\ für die drahtlose Kommunikation\\ zwischen Gebäuden und\\ Außeninstallationen} % Titel der Abschlussarbeit % no more than 4 lines here
\documentnumber{B\ 14-2015}%	MXX-20XX
\documenttype{Bachelorarbeit}%	Masterarbeit
\authorfirstname{Käpt'n Kevin}%	Name des Autors
\authorsurname{Blaubär}%		Nachname des Autors
\matrikelnumber{123456}%		Matrikelnummer des Autors
\date{23. Januar 2018}%			Abgabedatum

\makeatletter
\title{\@thesistitle}
\author{\@authorfirstname\space\@authorsurname}
\makeatother

%%%%%%%%%%%%%%%%%%%%%%%%%%%%%%%%%%%%%%%%%%%%%%%%%%%%%%%%%%%%%%%

\begin{document}
\frontmatter
\pagestyle{empty}
\maketitle
% \include{frontmatter/acknoledgement}
\pagestyle{headings}
% \include{frontmatter/abstract}
\tableofcontents%			Inhaltsverzeichnis

\mainmatter
\chapter{Einleitung} \label{cha:Einleitung}

In den letzten Jahren hat \gls{vlc} sowohl in der Wissenschaft als auch in der Industrie große Aufmerksamkeit erregt. Aufgrund seiner Vorteile hinsichtlich Bandbreitenverfügbarkeit, Sicherheit und Datensicherheit, \gls{vlc} wird in vielen Anwendungen als eine bessere Alternative zu Radiofrequenz und Infrarot betrachtet, wie z.B. drahtloses Netzwerk mit Beleuchtungssystem\cite{1205458}, Innenraumpositionierung \cite{4649677}, optische Verbindungen zu elektronischen Chips \cite{867694}, Körpersensornetzwerken \cite{bodysensor}, usw. Viele dieser \gls{vlc}-Techniken wurden entwickelt, um vorhandene Lichterzeugungsgeräte zu nutzen. Ein innovativer Ansatz ist die Verwendung von Display-Kamera-Paaren zur Datenübertragung. Aufgrund der steigenden Performance der beteiligten Schlüsselkomponenten erscheinen Datenraten von bis zu 100 Mbit/s realisierbar, während gleichzeitig eine Videopräsentation für menschliche Zuschauer auf dem gleichen Bildschirm zur Verfügung gestellt werden kann. Mit einem solchen System können viele innovative Medienanwendungen realisiert werden.

\section{Motivation} 

In einer aktuellen Forschung untersucht der Lehrstuhl für Kommunikationstechnik an der TU Dortmund ein neuartiges Verfahren der Visible Light Communication, d.h. \gls{david}-System, indem eine optische Freiraum-Übertragung unter Verwendung verfügbarer Anzeige und Kamera durchführen. Ein Highlight dieses Systems ist, dass sie die ursprüngliche Funktionalität des Displays nicht beeinträchtigt; Während der Datenübertragung kann das Display immer noch statische Bild oder Videos anzeigen, ohne dass menschliche Betrachter die versteckten Datensignale wahrnehmen. Die Übertragungsdaten werden einem Videosignal in Form geringer Amplitudenänderungen differentiell überlagert. Am Empfängen lassen sich der Display durch ein Kamera oder Smartphone aufnehmen. Anschließend können die überlagerte Daten empfangen und decodiert werden. Hierzu ist es notwendig, den Modulationsbereich, der durch die optische Projektion verzerrt wird, zu detektieren.

\section{Aufgabe in dieser Arbeit} 

In dieser Arbeit wird zwei Verfahren zur Ausschnittsdetektion für Differenzbilder untersucht und implementiert. Die erste Methode verwendet der Charakteristiken der Datenmodulation des \gls{david} Systems, d.h. auf Differenzbild wird das QR Muster detektiert, um das Modulationsbereich zu bestimmen. Die zweite Verfahren basiert auf der Geometrie des Displays. Mit Verwendung der Radon Transformation kann das rechteckig Modulationsbereich bestimmt werden. Darüber hinaus wird in dieser Arbeit eine Bildregistration Modul entwickelt, um die Einfluss, bei Aufnahme des Videos die Smartphone in der Handy gehalten, auszugleichen. Außerdem durch Einführung der Begriff $ ``Energie" $ wurde eine Optimierungsmethode für Differenzbild entwickelt. In Anschluss wurden die Performance und die Ersetzbarkeit der beiden Verfahren evaluiert. Der zweite Verfahren wurde auf einem Smartphone-GPU implementiert.

\section{Aufbau der Arbeit} 

Der Rest des Dokuments ist wie folgt organisiert.

Das nachfolgende Kapitel gibt einen Beschreibung über \gls{david} System. Es stellt die Struktur und Arbeitsweise des \gls{david} Systems vor und listet verschiedene Anwendungsbereiche auf, die möglicherweise vom \gls{david}-Konzept profitieren könnten.

Der dritte und vierte Absatz enthält Informationen zur beiden Verfahren. Die Struktur und Zusammensetzung der Methode werden vorgestellt und das Arbeitsprinzip jedes Teils wird detailliert beschreibt.

Das nächste fünfte Kapitel zeigt die Implementierung jeder Methode. Darin können die Effekt jedes Teils in der Methode sehen. %Schließlich zeigt die Implementierung auf eine Smartphone-GPU.

Kapitel 6 enthält eine Evaluierung der beiden Verfahren. Die Performance und die Einsetzbarkeit des Verfahrens werden analysiert.

Der letzte Abschnitt schließt die Ergebnisse ab und gibt einen Ausblick auf die zukünftige Arbeit.
\chapter{DaVid} \label{cha:DaVid}

In diesem Kapitel werden das \gls{david} System beschrieben. Zuerst läuft die Vorstellung des \gls{david} Systems. Die Systemmodell und Arbeitsprinzip des Systems werden in anschließenden Abschnitt erläutert. Schließlich folgt die mögliche Anwendungsgebiete des Systems. \cite{Kays2017,Kays2016,Kays201501,Kays201502}



%Durch diese Beschreibung wird die Bedeutung und Ziele dieses Papiers besser verstehen.\ldots

\section{Einführung des DaViD } 

\gls{david} ist ein neuartiges Verfahren zur optischen Freiraum-Datenübertragung zwischen einem Display als Sender und einer Kamera als Empfänger. Ein grundlegendes Übertragungskonzept von \gls{david} wird in Abbildung 2.1 gezeigt. Ein flaches Display wie ein OLED- oder LCD-Bildschirm zeigt ein Live-Video. Gleichzeitig werden die Daten hinter dem Bild auf die Pixel moduliert. Während die zusätzliche Datenmodulation für menschliche Betrachter nahezu unsichtbar ist, der Benutzer leitet ein hochauflösende Kamera oder ein Smartphone zur Bildschirm, um die Szene aufzunehmen. Durch der eingebaut Prozessor können die Signale decodiert werden.

\begin{figure}[htb]
 \centering 
 \includegraphics[keepaspectratio,width=0.8\textwidth]{images/2_DaViD/David1.jpg}
 \caption{Eine beispielhafte Implementierung des \gls{david}-Systems}
 \label{fig:David1}
\end{figure}


\section{Systemmodell} 
Bild in Display enthalten eine große Anzahl von Pixeln, die jeweils aus einer spezifischen Anordnung von Subpixeln für die RGB-Farbraum bestehen. Jeder einzelne Frame des Videos wird nämlich durch eine Matrix von Subpixelwerten dargestellt. \gls{david} System verwendet eine differentielle Modulationsmethod d.h. Teil der Videoinformationen muss wiederholt werden, indem Daten als ein symmetrischer Manchester-Code moduliert und zu den Videosignalkomponenten hinzugefügt werden. In Empfängerseite durch eine zeitliche Synchronisation können die zeitliche \gls{ISI} vermieden werden. Dann nach Verwendung einer örtlichen Synchronisation enthalten einen Differenzbild. Weil die Randbereich des Differenzbilds ungültig ist, verlässt sich die Modulationsgebiet durch die Verfahren in diese Arbeit entdecken. Danach werden die überlagerten Datensequenz durch eine Reihe von Behandlungen vom Videoinhalt getrennt. Abbildung 2.2 zeigt die schematische Darstellung des \gls{david}-Systems. 
% Deshalb in zeitlicher oder örtlicher Richtung die Videoinhalt in paar Bildern werden gleich.
\vspace{18pt}

\begin{figure}[htb]
	\centering 
	\includegraphics[keepaspectratio,width=1.0\textwidth]{images/2_DaViD/David3.jpg}
	\caption{Schematische Darstellung von \gls{david} System}
	\label{fig:David2}
\end{figure}


\subsection{Modulationsverfahren}

Ein Modulationsverfahren, das die Videoqualität nicht offensichtlich reduziert garantiert, ist sehr wichtig für ein auf Videogerät basierendes Datenübertragungssystem. Die möglichen Modulationsverfahren in \gls{david}-System sind:
\begin{itemize}
	\item Zeitliche differentielle Modulation der Luminanz
	\item Zeitliche differentielle Modulation der Chrominanz
	\item Örtliche differentielle Modulation der Luminanz
	\item Örtliche differentielle Modulation der Chrominanz
\end{itemize}

Zeitliche differentielle Modulationsverfahre lässt kontinuierliches Paar Frames den gleichen Luminanz- bzw. Chrominanz-Videoinhalt enthalten, d.h. durch Subtrahieren die mit daten addiert Kanal der Paar Frames die Differenzbild erhalten lassen können. Dagegen in örtliche- sind die benachbarte Pixel mit den gleichen Videoamplituden. Hier wird in dieser Arbeit nur zeitliche differentielle Modulation der Chrominanz verwendet. Abbildung 2.3 zeigt ein Blockdiagramm einer typischen Senderimplementierung durch zeitliche differentielle Modulation.

\begin{figure}[htb]
	\centering 
	\includegraphics[keepaspectratio,width=0.8\textwidth]{images/2_DaViD/David2.jpg}
	\caption{Blockschaltbild der Signalverarbeitung in zeitlicher differentieller Modulation}
	\label{fig:David3}
\end{figure}

Wir nehmen eine Diaplay an, die in horizontale Richtung $N_x$ Pixel stehen, dagegen in vertikale Richtung $N_y$ Pixel. Das Videoeingangssignal wird verarbeitet, um eine Anzeigeeingabe $s(i,j,k)$ zu liefern. Mit zeitliche differenziell Modulation ist die Videoinhalt des kommenden Frame dasselbe.Die Indizes i und j bedeuten die horizontale und vertikale Pixelposition auf dem Bildschirm, während k die Nummer des reproduzierten Bildes ist.
Indiz m heißt den Zähler des Frames in einer Videosequenz. Der Amplitudenbereich des Videosignals sollte begrenzt sein, um die Addition kleiner Datenamplituden ohne Übersteuern zu ermöglichen.

\begin{equation}
\begin{split}
 s_{R/G/B}&(i,j,k+1) = s(i,j,k) \\
          &  for \ 0\le i <N_x, 0\le j<N_y,k=2 \cdot m , m \in \mathbb{Z}\\
\end{split}
\end{equation}

Vor Datenübertragung muss der Datenstrom in Schichten der Länge L aufgeteilt werden. Indiz L bedeutet die Menge der Daten, die in einem Framepaar übertragen werden können. Ein direkter Ansatz ist eine direkte Zuordnung von Datenbits zu Pixeltripeln Zeile für Zeile.

\begin{equation}
\begin{split}
  & d(l)\rightarrow d(i,j,m) \qquad d(l)\in \{-1,1\} \\
  & 0\le l <L,L = N_x \cdot N_y \\
  & i=l \bmod N_x \qquad j=\lfloor l/N_x \rfloor \\
\end{split}
\end{equation}

Die Modulationsamplitude A ist ein wichtiger Parameter für Datenübertragung. Im Prinzip kann die Amplitude in verschieden Kanal unabhängig gewählt werden, um die Systemleistung zu optimieren. In diesen Arbeiten setzen die Amplitude gleichwertig.

\begin{equation}
 A_R=A_G=A_B=A        
\end{equation}

Das differentielle Modulationsverfahren ordnet jede Sequenz von $\left\{-A, A\right\}$ zu d = -1 bzw. $\left\{A, -A\right\}$ zu d = 1 zu. Modulierte Datensymbole und verarbeitete Videoamplituden werden addiert, um die Anzeigeeingabe $g(i,j,k)$ zu liefern:

\begin{equation}
\begin{split}
   s_{R/G/B}(i,j,k)  &= s_{R/G/B}(i,j,m) + A_{R/G/B} \cdot \left( 2 \cdot d(i,j,m) - 1 \right) \\
   s_{R/G/B}(i,j,k+1)&= s_{R/G/B}(i,j,m) - A_{R/G/B} \cdot \left( 2 \cdot d(i,j,m) - 1 \right) \\
\end{split}
\end{equation}

Ein Beispiel einer modulierten Bildfolge ist in Abbildung 2.4 gezeigt. Das Hinzufügen der modulierten Daten (hier mit A = 4) zu dem Videoeingang ergibt die Anzeigeamplituden in der rechten Spalte.
\newpage

\begin{figure}[htb]
	\centering 
	\includegraphics[keepaspectratio,width=0.6\textwidth]{images/2_DaViD/David4.jpg}
	\caption{Ein Beispiel einer modulierten Bildfolge}
	\label{fig:David4}
\end{figure}

Im Vergleich zu Luminanzteil Y die Anzeigequalität in U und V Komponente ist signifikant besser, wenn Informationen in Chrominanz wiederholt und moduliert werden. Auf diese Weise wird die Gesamtleuchtdichte eines Pixel-Triple durch die Datenmodulation nicht beeinflusst.Die Umwandlungsmatrix, angegeben von \gls{ITU-R BT.709} für \gls{HDTV} Display von Standard $(R,G,B)$ zur Standard $(Y,U,V)$ läuft:

\begin{equation}
   T = \begin{pmatrix}
   0,213 & 0,715 & 0,072 \\
   -0,115& -0,385& 0,5	\\
   0,5   & -0,454& -0,0458
\end{pmatrix}  
\end{equation}

Das Videosignal $s(i,j,k)$ muss vor dem Anwenden der Modulation in Y-, U- und V-Komponenten umgewandelt werden. Die nachfolgende inverse Konvertierung erklärt das Display-Eingangssignal in Abbildung 2.3:

\begin{equation}
\begin{split}
  \begin{pmatrix}
  g_R(i,j,k) \\
  g_G(i,j,k) \\
  g_B(i,j,k) \\
\end{pmatrix} &= T^{-1} \cdot \left( T \cdot \begin{pmatrix}
  S_R(i,j,m) \\
  S_G(i,j,m) \\
  S_B(i,j,m) 
  \end{pmatrix}  + \begin{pmatrix}
  0 \\
  A_U \cdot d(i,j,m) \\
  A_V \cdot d(i,j,m) 
  \end{pmatrix} \right) \\  
  \begin{pmatrix}
  g_R(i,j,k+1) \\
  g_G(i,j,k+1) \\
  g_B(i,j,k+1) \\
\end{pmatrix} &= T^{-1} \cdot \left( T \cdot \begin{pmatrix}
  S_R(i,j,m) \\
  S_G(i,j,m) \\
  S_B(i,j,m) 
  \end{pmatrix}  - \begin{pmatrix}
  0 \\
  A_U \cdot d(i,j,m) \\
  A_V \cdot d(i,j,m) 
  \end{pmatrix} \right) \\ 
\end{split}
\end{equation}

Diese Art der Modulation kann als eine Modulation des roten und des blauen Subpixels betrachtet werden, während das grüne Subpixel verwendet wird, um die Änderung der Luminanz des Pixel-Tripels zu kompensieren. Durch Definition korreliert $A_U$ mit $A_B$ und $A_V$ mit $A_R$.

\begin{equation}
   \begin{pmatrix}
   A_R \\
   A_G \\
   A_B
  \end{pmatrix}  = T^{-1} \cdot \begin{pmatrix}
   A_Y \\
   A_U \\
   A_V
  \end{pmatrix}
\end{equation}



\subsection{DatenBlock}

Um die Anforderungen an die Kameraauflösung zu lockern, Ein einfaches und unkompliziertes Verfahren ist Zuordnung jedes Datenbits zu einem Block von $B_X \times B_Y$ Pixeln.

\begin{equation}
\begin{split}
  & d(l)\rightarrow d(x,y,k) \qquad 0\le l <L \\
  & L=\lfloor N_X/B_X \rfloor \cdot \lfloor N_Y/B_Y \rfloor \\
  & x=(l \cdot B_X) \bmod N_X +r_X, \ r_X =0...(B_X -1) \\
  & y=\lfloor l / \lfloor N_X/B_X \rfloor \rfloor \cdot B_Y +r_Y, \ r_Y =0...(B_Y -1) \\
\end{split}
\end{equation}

Wenn die Anzahl der Pixel pro Zeile kein Vielfaches von $B_X$ ist oder wenn die Anzahl der Pixel kein Vielfaches von $B_Y$ ist, muss die Anzahl der Pixel und Zeilen, die für die Modulation in Gleichung $\left(2.5\right)$ verwendet werden, ersetzt werden durch:

\begin{equation}
\begin{split}
  & N_X = \lfloor N_X/B_X \rfloor \cdot B_X\\ 
  & N_Y = \lfloor N_Y/B_Y \rfloor \cdot B_Y\\ 
\end{split}
\end{equation}

% \newpage
In dieser Arbeit werden das DatenBlock für quadratische Blöcke gesetzt.

\begin{equation}
   B_X = B_Y = B.
\end{equation}


\section{Anwendungsgebiete} 
%\label{sec:Anwendungsbereiche}

Die Datenübertragungsrate des \gls{david}-Systems wird voraussichtlich erreicht bis zu 100 Mbit/s. Es gehöre	zu einer Sichtlinienübertragung für kurze Verbindungen. Geeignete Abdeckungsbereichen hängen von der Größe des Displays und der Kameraoptik ab. Obwohl im Vergleich zum letzten WLAN- Versionen \gls{ieee} 802.11, die Leistung scheint nicht so attraktiv. Der Vorteil liegt nicht nur in der wachsenden Leistungsfähigkeit von Video-Display und Kamera, aber auch die Option zur Wiederverwendung der bestehenden Hardware, die zum Zeigen des Videos installiert wurde. Ein empfohlene praktische Anwendungsbereich des \gls{david}s ist öffentlicher Ort, z.B. U-Bahn-Station, großes Stadion und so weiter. Annehmen eine Situation, wenn die Leute auf ihre U-Bahn warten, sie können ihre eigene Software aktualisieren, indem Sie einfach auf die zeigende Werbung in dem Bildschirm leiten.

Berücksichtigen der Eigenschaften des \gls{david}s, d.h. die Synchronisation von Videospielen und Datenübertragung. Viele Anwendungsszenarien können in Betracht gezogen werden und scheinen sehr attraktiv zu sein. Die drei Hauptszenarien sind:

\begin{itemize}
  \item Indoor-individuelle Kommunikation: Kurzstreckenverbindungen basieren auf relativ kleinen (Tablet-Größe) Bildschirm, Anwendungen z.B. die Übertragung von Hintergrundinformationen an Besucher im Museum, Kiosk.
  \item Indoor-Multicast-Kommunikation:Streckabstand ist länger als ersten Fall auf relativ größer (40-100") Bildschirm, Anwendungen z.B. während Videoabspielen Besucher die Anwendungsdaten oder Mediendateien herunterladen können im Kiosk, Restaurant.
  \item Freien Kommunikation: Größter Bildschirm wie im Einkaufszentren oder Sport-Arenen, Anwendungen können denen des zweiten Szenarios ähneln.
\end{itemize}

Sobald die Dienste auf öffentlichen Bildschirmen implementiert werden, kann Leute mit Hilfe eines modernen Smartphones, die mit einer geeigneten Kamera eingebaut ist, nach der Installation einer neuen App innovative wahrnehmen.






















\chapter{Implementierung} \label{cha:Implementierung}

Nicht vergessen, dass Überschriften nicht aufeinander folgen dürfen\ldots

\begin{otherlanguage}{english}
\section{TexLipse spell checking}
%
To enable spell checking in TeXLipse, download the respective dictionaries from 
\url{https://sourceforge.net/projects/texlipse/files/dictionaries/}.

Save the dictionaries at a local location and enter the path in \texttt{Window->Preferences->Tex\-lipse->Spell Checker} (see Fig. \ref{fig:dict_path}).
%
\begin{figure}[htb]
	\centering
	\includegraphics[scale=0.40]{images/Spell_Checker_preferences.jpg}
	\captionbelow{TeXLipse Spell Checker preferences}
	\label{fig:dict_path}
\end{figure}

To synchronize the user dictionaries between multiple machines, it might be useful to save the dictionaries in your google drive or drop box.

\section{Enable tikzexternalize for PdfLatex}

Go to \texttt{Window->Preferences->Texlipse->Builder Settings} and add 
%
\begin{verbatim}
--shell-escape
\end{verbatim}
%
to the command arguments (see Fig. \ref{fig:builder_settings}).
%
\begin{figure}[htb]
	\centering
	\includegraphics[scale=0.40]{images/PdfLatex_settings.jpg}
	\captionbelow{PdfLatex Builder Settings}
	\label{fig:builder_settings}
\end{figure}

\section{Forward search with TeXlipse and Sumatra PDF}

Download and install SumatraPDF: \url{https://www.sumatrapdfreader.org/}.

Then edit the viewer settings for SumatraPDF in \texttt{Window->Preferences->Texlipse->Viewer Settings}.

Change the viewer arguments to
%
\begin{verbatim}
-reuse-instance %fullfile -forward-search %texfile %line
\end{verbatim}
%
and leave all DDE message field empty.
Change the inverse search support to "`Viewer runs external command"' and enable "`Viewer supports forward search"'.

Figure \ref{fig:viewer_settings} displays the dialog window.
%
\begin{figure}[htb]
	\centering
	\includegraphics[scale=0.40]{images/Viewer_settings.jpg}
	\captionbelow{TeXLipse Viewer Settings}
	\label{fig:viewer_settings}
\end{figure}

In SumatraPDF configure the inverse search command via the \texttt{Settings->Options} menu (see Fig. \ref{fig:sumatrapdf_options}).
%
\begin{figure}[htb]
	\centering
	\includegraphics[scale=0.40]{images/SumatraPDF_optionen.jpg}
	\captionbelow{SumatraPDF Options}
	\label{fig:sumatrapdf_options}
\end{figure}

If you have install TeXlipse~1.5.0, the inverse search command will look like this:

\begin{lstlisting}[breaklines=true, basicstyle=\ttfamily, columns=flexible]
javaw -classpath "C:\Users\wu\.p2\pool\plugins\net.sourceforge.texlipse_1.5.0\texlipse.jar" net.sourceforge.texlipse.viewer.util.FileLocationClient -p 55000 -f "%f" -l %l
\end{lstlisting}

Let the path point to your eclipse share pool. Or if you do not have a shared pool, choose the plugins directory of your eclipse installation.

For TeXLipse~2.0.X the FileLocationClient is relocated to org.eclipse.texlipse making the inverse search command look like the following.
\begin{lstlisting}[breaklines=true, basicstyle=\ttfamily, columns=flexible]
javaw -classpath "C:\Users\wu\.p2\pool\plugins\org.eclipse.texlipse_2.0.1.201801202105\texlipse.jar" org.eclipse.texlipse.viewer.util.FileLocationClient -p 55000 -f "%f" -l %l
\end{lstlisting}

\end{otherlanguage}
\chapter{Auswertung} \label{cha:Auswertung}

Text hier zwischen. Referenz auf ein Bild mit cleverref (siehe \cref{fig:mathplot}). Und dann noch ein paar Zitierungen \cite{Reinhold:2013fk,Moon,IEEE2011}.

\section{Section}

Jetzt nur noch schreiben! :)



\chapter{Zusammenfassung} \label{cha:Zusammenfassung}

Nicht vergessen, dass Überschriften nicht aufeinander folgen dürfen\ldots


\section{Section}

\appendix
\chapter{Erster Anhang} \label{cha:anhangA}

\section{Section}

Jetzt nur noch schreiben!  I am a child.





\backmatter
\cleardoublepage
% \chapter*{Nomenklatur} \addcontentsline{toc}{chapter}{Nomenklatur}
% \markboth{Nomenklatur}{Nomenklatur}

\printglossary[type=\acronymtype, style=tab_style] %[toctitle=Symbolverzeichnis,title=Symbolverzeichnis] 
%\printglossary[type=symbol, style=tab_style_sym]
% \printglossary

\listoffigures%				Abbildungsverzeichnis
\listoftables%				Tabellenverzeichnis
\lstlistoflistings%			Listingsverzeichnis

% \nocite{*}%				es wird alles zitiert
\printbibliography%			Literaturverzeichnis

\makeatletter
\cleardoublepage
\thispagestyle{empty}
\onehalfspacing
\LARGE
\begin{center}
	\textbf{Eidesstattliche Versicherung}
\end{center}

\small
\vspace{\baselineskip}
\underline{\makebox[.4\textwidth][c]{\@authorsurname,\space\@authorfirstname}}
\hspace{.25\textwidth}
\underline{\makebox[.3\textwidth][c]{\@matrikelnumber}}
\newline
\makebox[.4\textwidth][l]{Name, Vorname}
\hspace{.25\textwidth}
\makebox[.3\textwidth][l]{Matr.-Nr.}

Ich versichere hiermit an Eides statt, dass ich die vorliegende \@documenttype\space mit dem Titel

\underline{\makebox[0.97\textwidth][c]{\@thesistitle}}

selbstständig und ohne unzulässige fremde Hilfe erbracht habe. Ich habe keine anderen als die 
angegebenen Quellen und Hilfsmittel benutzt sowie wörtliche und sinngemäße Zitate kenntlich 
gemacht. Die Arbeit hat in gleicher oder ähnlicher Form noch keiner Prü\-fungs\-be\-hörde 
vorgelegen. 
\vspace{\baselineskip}

\underline{\makebox[.4\textwidth][c]{Dortmund, \today}}
\hspace{.25\textwidth}
\underline{\hspace{.3\textwidth}}
\newline
\makebox[.4\textwidth][l]{Ort, Datum}
\hspace{.25\textwidth}
\makebox[.3\textwidth][l]{Unterschrift}

\vspace{\baselineskip}

\textbf{Belehrung}: 

Wer vorsätzlich gegen eine die Täuschung über Prüfungsleistungen betreffende Regelung einer 
Hochschulprüfungsordnung verstößt, handelt ordnungswidrig. Die Ordnungswidrigkeit kann mit 
einer Geldbuße von bis zu 50.000,00\,€ geahndet werden. Zuständige Verwaltungs\-behörde für 
die Verfolgung und Ahndung von Ordnungswidrigkeiten ist der Kanzler/die Kanzlerin der 
Technischen Universität Dortmund. Im Falle eines mehrfachen oder sonstigen schwerwiegenden 
Täuschungs\-versuches kann der Prüfling zudem exmatrikuliert werden. (§ 63 Abs. 5 
Hochschulgesetz - HG - )  

Die Abgabe einer falschen Versicherung an Eides statt wird mit Freiheitsstrafe bis zu 3 Jahren 
oder mit Geldstrafe bestraft.  

Die Technische Universität Dortmund wird ggf. elektronische Vergleichswerkzeuge (wie z.B. die 
Software „turnitin“) zur Überprüfung von Ordnungswidrigkeiten in Prüfungsverfahren nutzen. 

Die oben stehende Belehrung habe ich zur Kenntnis genommen: 
\vspace{\baselineskip}

\underline{\makebox[.4\textwidth][c]{Dortmund, \today}}
\hspace{.25\textwidth}
\underline{\hspace{.3\textwidth}}
\newline
\makebox[.4\textwidth][l]{Ort, Datum}
\hspace{.25\textwidth}
\makebox[.3\textwidth][l]{Unterschrift}
\makeatother
\end{document}